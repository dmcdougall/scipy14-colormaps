\documentclass[10pt,presentation,compress]{beamer}

\usepackage{graphicx}
\usepackage{parskip,amssymb,amsmath,amsthm,url}
\usepackage{subfigure}
\usepackage{esdiff}
\usepackage{latexsym}
\usepackage{color}

\title{How to choose a good colour map}
\author{Damon McDougall}
\institute{Institute for Computational and Engineering Sciences, UT Austin, USA}
\date{10th July 2014}

\mode<presentation> {
  \usetheme{Singapore}
  \setbeamertemplate{navigation symbols}{}
  \setbeamertemplate{mini frames}{}
  \setbeamertemplate{section in head/foot}{}
  \setbeamercovered{transparent=0}
  \setbeameroption{show notes}
  \setbeamertemplate{footline}
  {
    \leavevmode
    \hbox{
     \begin{beamercolorbox}[wd=.99\paperwidth,ht=2.5ex,dp=1.125ex,right]{title in head/foot}
        \usebeamerfont{}\insertframenumber/\inserttotalframenumber
      \end{beamercolorbox}
    }
    \vskip 0pt
  }
}
\mode<handout> {
  \usetheme{Rochester}
  \usepackage{pgfpages}
  \pgfpagesuselayout{4 on 1}[a4paper,landscape,border shrink=5mm]
  \setbeamertemplate{background canvas}{bg=black!5}
  \setbeamercolor{normal text}{fg=black}
}

\newcommand{\ud}{\,\mathrm{d}}
\newcommand{\grad}{\nabla}
\newcommand{\Var}{\mbox{Var}}
\newcommand{\Cov}{\mbox{Cov}}
\newcommand{\argmin}{\mbox{argmin}}

\newcommand{\iid}{
  \ensuremath{
    \stackrel{\mbox{\scriptsize{i.i.d}}}{\sim}
  }
}

\newcommand{\norm}[1]{
  \ensuremath{\left\| #1 \right\|}
}

\begin{document}
\begin{frame}
  \titlepage
\end{frame}

\begin{frame}{Introduction}
  Most of the content is taken from this excellent article:
  \linebreak
  \linebreak
  \textcolor{blue}{\small{\url{http://www.research.ibm.com/people/l/lloydt/color/color.HTM}}}
\end{frame}

\begin{frame}{Introduction}
  \begin{itemize}
    \item Data is a huge aspect of science
    \item By and large we (scientists) treat data well\dots
    \item \dots and we visualise it poorly.  Why?
    \item Colour maps
    \item Data is of some field
      $f : \Omega \subset \mathbb{R}^2 \to [0, 1]$
    \item A colour is assigned to the output of $f$ (a scalar).  Seems
      reasonable.
    \item Colour map is a function
      $g : [0, 1] \to \Omega' \subset \mathbb{R}^3$
    \item Mismatch in dimensions:  $\mathbb{R}^3$ versus $\mathbb{R}$
    \item The point?  Colour maps can be misleading.
  \end{itemize}
\end{frame}

\begin{frame}{Hating on the jet colour map}
  What is this?
  \begin{figure}[htp]
    \includegraphics[scale=7.0]{florida_masked.jpg}
  \end{figure}
\end{frame}

\begin{frame}{Hating on the jet colour map}
  Did anybody see Florida?
  \begin{figure}[htp]
    \includegraphics[scale=7.0]{florida.jpg}
  \end{figure}
  \begin{itemize}
    \item Left: Linear interpolation in RGB space between red and blue.
    \item Right: Changes in data are \textit{perceived} as proportional
      changes in colour (subjective)
    \item Right: Domain specific knowledge used to reveal important features
  \end{itemize}
\end{frame}

\begin{frame}{Hating on the jet colour map}
  \begin{figure}[htp]
    \includegraphics[scale=0.75]{chesapeake.jpg}
  \end{figure}
\end{frame}

\begin{frame}{Hating on the jet colour map}
  \begin{figure}[htp]
    \includegraphics[scale=0.75]{mri.jpg}
  \end{figure}
\end{frame}

\begin{frame}{Hating on the jet colour map}
  \begin{figure}[htp]
    \includegraphics[scale=0.75]{jetnoise.jpg}
  \end{figure}
\end{frame}

\begin{frame}{Hating on the jet colour map}
  \begin{figure}[htp]
    \includegraphics[scale=0.75]{magfield.jpg}
  \end{figure}
\end{frame}

\begin{frame}{Hating on the jet colour map}
  \begin{figure}[htp]
    \includegraphics[scale=0.75]{sinc.jpg}
  \end{figure}
\end{frame}

\begin{frame}{What have we learned?}
  \begin{itemize}
    \item Jet is not a great colourmap (or is it?)
    \item Two types of information one can glean from a colourmap\footnote{C.
      Ware, Color sequences for univariate maps: theory, experiments, and
      principles, IEEE Computer Graphics and Appliations, 1998.}
      \begin{itemize}
        \item `Value' or `metric' information
        \item `Form' or `structure' information
      \end{itemize}
    \item Jet is not bad for value information (but not everywhere)
    \item Jet is awful for form information
    \item Jet is not alone---but it is very commonly used (see K Thyng's talk)
    \item How to pick a good colour map?  \textbf{It depends!}
  \end{itemize}
\end{frame}

\begin{frame}{What is good for form/structure information?}
  \begin{itemize}
    \item Colour has 3 dimensions: hue, saturation, and luminance
    \item Saturation-varying colourmaps are good for low-frequency data
    \item Luminance-varying colourmaps are good for high-frequency data
    \item The human brain is very bad an interpolating hue\footnote{Conclusion
      from psychophysical experiments by S. S. Stevens (formerly at Harvard)}
    \item Perceptually-based colourmaps
      \begin{itemize}
        \item Equal steps in data are perceived as equal steps in the colour
          space
      \end{itemize}
  \end{itemize}
\end{frame}

\begin{frame}{Perceptually-based colourmaps}
  % Top-left is jet
  % Top-right is perceptual for HF
  % Bottom-left is perceptual for LF
  % Bottom-right tried to encompass both HF and LF
  \begin{figure}[htp]
    \includegraphics[scale=7.0]{perceptual1.jpg}
  \end{figure}
\end{frame}

\begin{frame}{Perceptually-based colourmaps}
  \begin{figure}[htp]
    \includegraphics[scale=0.55]{pchesapeake.png}
  \end{figure}
\end{frame}

\begin{frame}{Perceptually-based colourmaps}
  \begin{figure}[htp]
    \includegraphics[scale=0.55]{pmri.png}
  \end{figure}
\end{frame}

\begin{frame}{Perceptually-based colourmaps}
  \begin{figure}[htp]
    \includegraphics[scale=0.55]{pjetnoise.png}
  \end{figure}
\end{frame}

\begin{frame}{Perceptually-based colourmaps}
  \begin{figure}[htp]
    \includegraphics[scale=0.55]{pmagfield.png}
  \end{figure}
\end{frame}

\begin{frame}{Perceptually-based colourmaps}
  \begin{figure}[htp]
    \includegraphics[scale=0.55]{psinc.png}
  \end{figure}
\end{frame}

\begin{frame}
  \begin{center}
    Thank you.
    \linebreak
    \linebreak
    Pssst.  Jet is the default colourmap in matplotlib.  Anybody want to fix it?  Submit a PR!
  \end{center}
\end{frame}

\end{document}
